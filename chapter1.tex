\chapter{Introduction}

\begin{bibunit}

The discovery of the cosmic microwave background (CMB) radiation by \citet{1965ApJ...142..419P}
provided the first direct evidence that the universe had a beginning. Arno Penzias and Robert Wilson
shared the 1978 Nobel Prize in Physics for this discovery, and astronomers have been studying this
radiation ever since. In fact, a second Nobel Prize was awarded to John Mather and George Smoot in
2006 for their work on the Cosmic Background Explorer (COBE) satellite, which was amongst the first
experiments to demonstrate that the background radiation was anisotropic
\citep{1992ApJ...396L...1S}. These studies of the CMB have fundamentally advanced humanity's
understanding of the universe: its origin, evolution, and composition. Still we continue to study
the CMB particularly because it illuminates everything in the universe. It is a flashlight for the
darkness of space within our expanding universe.

As the universe expands, the wavelength of a photon is similarly stretched or redshifted (so-called
because it gradually drifts to longer, redder wavelengths). Photons originating from a star 1000
light-years away will travel through the universe for 1000 years before they are collected by our
telescopes. Consequently we observe this star as it was 1000 years ago. However, during its travels,
the photon was also stretched by a small factor of $0.000007\%$ due to the expansion of the
universe.  For nearby stars, this expansion factor is clearly too small to be conceivably measured.
However, with the discovery of the first quasar by \citet{1963Natur.197.1040S} it soon became
apparent that the stretching factor, the redshift $z$, can be $>10\%$. Today, the most distant known
galaxies are so far away that the wavelength has more than doubled ($z > 1$) due to the expansion of
the universe.

list some known galaxies and their redshifts

Due largely to careful and detailed work studying the CMB, Type Ia supernova explosions, and
cosmological galaxy surveys, we have a very detailed understanding of the expansion history of the
universe.

Given knowledge of the
original wavelength of the photon, and the expansion history of the universe, we can calculate how
long the photon must have been in flight.


Today the CMB is a 2.7~K sea of photons that permeates the universe. This radiation is constantly
cooling due to the inexorable expansion of the universe.


Introduce the 21 cm transition.

Introduce low frequency telescopes.












\myputbib{thesis}
\end{bibunit}

