\cleartoevenpage

\myepigraph
{This is radio astronomy!}
{Tony Readhead}

\chapter{A Path Towards Calibration of the OVRO-LWA}
\label{chapter2}

\begin{bibunit}

\section{Design and Construction of the OVRO-LWA}

\begin{figure}
    \centering
    \includegraphics[width=\textwidth]{figures/chapter2/first-antenna}
    \caption{
        The first OVRO-LWA completed on 2013 March 8 with the class of Ay~122b (including the author
        of this thesis with the fractured ankle).
    }
    \label{fig:ovro-first-antenna}
\end{figure}

\begin{figure}
    \centering
    \includegraphics[width=\textwidth]{figures/chapter2/lwa-antenna}
    \caption{
        Picture of an OVRO-LWA antenna.
    }
    \label{fig:ovro-lwa-pictures}
\end{figure}

The OVRO-LWA is a new low-frequency (27--85\,MHz) radio interferometer constructed during the course
of this thesis and located near Big Pine, California. Construction began in 2013 with the first
antenna completed in March of that year (see Figure~\ref{fig:ovro-first-antenna}).

The OVRO-LWA was initially composed of 256 antennas, with 251 of those antennas arranged within a
dense 200\,m diameter core in a configuration that is optimized for sidelobe levels in snapshot
imaging.  Each of these antennas consists of two crossed broadband dipoles with an active
balun/preamp, and the entire system is sky noise dominated over the range 20--80\,MHz
\citep{2012PASP..124.1090H}.  A picture of an OVRO-LWA antenna can be seen in
Figure~\ref{fig:ovro-lwa-pictures}. The primary beam of each OVRO-LWA antenna subtends a solid angle
of $\sim 8000\,\text{deg}^2$, with sensitivity to the entire visible hemisphere of the sky.

%\todo{ARX boards}

The remaining five antennas are isolated from the core of the OVRO-LWA and equipped with radiometric
front ends as part of the LEDA experiment, which is attempting to measure the globally averaged
signal of \ion{H}{1} from the Cosmic Dawn \citep{2018MNRAS.478.4193P}.  The OVRO-LWA hosts the LEDA
correlator as its back-end, which is an FX correlator composed of 16 ROACH2 FPGAs that form the F
stage, and 11 servers each with dual NVIDIA K20X GPUs that form the X stage
\citep{2015JAI.....450003K}. This allows the correlator to perform full cross-correlation of 512
input signals with 58\,MHz instantaneous bandwidth.

During observations, data is streamed from the LEDA correlator to the All-Sky Transient Monitor
(ASTM), which houses the compute nodes used for post-processing and imaging.  The ASTM is composed
of 10 identical nodes each with a 16-core Intel Xeon E5-2630 v3 CPU and 64\,GB of memory. Five
additional servers provide 565\,TB of storage capacity through the Lustre high performance file
system.

The initial 256 antenna interferometer was therefore capable of imaging the entire visible
hemisphere in snapshot images with 1$\arcdeg$ angular resolution. An example snapshot image can be
seen in Figure~\ref{fig:core-snapshot-image}.  In 2015, an additional 32 antennas were installed
that extended the maximum baseline of the interferometer to 1.5\,km and improved the angular
resolution to 8$\arcmin$. This improved angular resolution can be seen in
Figure~\ref{fig:expansion-snapshot-image}. In a final future stage of development, an additional 64
antennas will be installed that extend the maximum baseline to 2.6\,km, which will see improved
$uv$-coverage at long baselines and improved $5\arcmin$ angular resolution, as well as an expanded
correlator to accommodate the additional antennas.

While this thesis focuses on the use of the OVRO-LWA to detect the high-redshift signature of
neutral hydrogen from the Cosmic Dawn, the OVRO-LWA facilitates a diverse set of scientific
motivations including the study of stellar and planetary magnetospheres, radio follow-up of neutron
star mergers and gamma ray bursts \citep{2017arXiv171106665A}, solar dynamic imaging spectroscopy,
and the detection of high energy cosmic rays \citep{caltechthesis11016}.

With the exception of detecting high energy cosmic rays---which uses a custom firmware and
processing pipeline that currently cannot operate in parallel with ordinary correlation---there are
two complementary software pipelines that service the scientific goals of the OVRO-LWA:
\begin{enumerate}
    \item A widefield snapshot imaging pipeline that images the entire visible hemisphere every
        13\,s using \texttt{WSCLEAN} \citep{2014MNRAS.444..606O}.
    \item A novel approach, called $m$-mode analysis, that is specialized for drift-scanning
        interferometers that can image the entire sky (above a limiting declination) in a single
        synthesis imaging step.
\end{enumerate}
This chapter will describe the calibration and source removal routines purpose built for the
OVRO-LWA and used by both pipelines. Additionally, in \S\ref{sec:commissioning-challenges} I will
discuss some of the challenges involved with commissioning the OVRO-LWA that have been overcome to
bring the OVRO-LWA into existence, to first light, and finally its first scientific results.
Finally, the latter pipeline will be discussed in considerable depth in Chapters~\ref{chapter3} and
\ref{chapter4}.

\begin{figure}[p]
    \centering
    \includegraphics[width=\textwidth]{figures/chapter2/before-expansion}
    \caption{
        Snapshot image of the sky captured with the OVRO-LWA and using only the antennas located
        within the core of the array. The image covers the entire visible hemisphere in
        sine-projection.  A similar image constructed using the newer long-baseline antennas can be
        seen in Figure~\ref{fig:expansion-snapshot-image}.
    }
    \label{fig:core-snapshot-image}
\end{figure}

\begin{figure}[p]
    \centering
    \includegraphics[width=\textwidth]{figures/chapter2/after-expansion}
    \caption{
        Snapshot image of the sky captured with the OVRO-LWA and using the new long-baseline
        antennas. The image covers the entire visible hemisphere in sine-projection.  A similar
        image constructed using only the core of the interferometer can be seen in
        Figure~\ref{fig:core-snapshot-image}.
    }
    \label{fig:expansion-snapshot-image}
\end{figure}

\section{Calibration of a Low-Frequency Interferometer}\label{sec:gain-calibration}

The purpose of calibration is to remove the contribution of the antenna and receivers, including any
gain, filters, and propagation effects along the signal path. At low radio frequencies, the Earth's
ionosphere is additionally important due to the effects of electromagnetic waves propagating through
a magnetized plasma.

We will make a distinction between a ``direction-independent'' calibration and a
``direction-dependent'' calibration. The former can correct for the response of the receiver and
signal path, but cannot fully account for the antenna response pattern and ionospheric effects. The
latter allows for the calibration parameters to vary as a function of direction on the sky, which
can account for errors due to the antenna beam and some ionospheric effects.

Neglecting complexity associated with polarized imaging, a direction-independent calibration amounts
to determining the complex-valued gain $g_i(\nu)$ associated with each signal path $i$ and frequency
$\nu$. These gains effect the measured correlation between the signal paths $i$ and $j$ such that
\begin{equation}
    V_{ij}^\text{measured}(\nu) = g_i(\nu)\,g_j^*(\nu)\,V_{ij}^\text{true}(\nu)
        + \text{noise}\,,
\end{equation}
where $V_{ij}^\text{measured}$ is the visibility actually measured between the corresponding signal
paths at the frequency $\nu$, and $V_{ij}^\text{true}$ is the visibility that would have been
measured if instead we had correlated the value of the electric field at the electrical center of
each antenna without the need for any additional electronics. The true visibility
can be computed from the sky brightness $I(\nu, \hat{r})$ at the
frequency $\nu$ and direction $\hat r$ such that
\begin{equation}
    V_{ij}^\text{true}(\nu) = \int a_i(\nu, \hat{r})\,a_j^*(\nu, \hat{r})\,I(\nu, \hat{r})\,
        \exp\Big(2\pi i\hat{r}\cdot\vec{b_{ij}}/\lambda\Big)\,\d\Omega,
\end{equation}
where the integral runs over solid angle $\Omega$, $a_i(\nu, \hat{r})$ is the response of the
corresponding antenna at the frequency $\nu$ to the direction $\hat{r}$, $\vec{b_{ij}}$ is the
baseline separating the antennas for signal paths $i$ and $j$, and $\lambda$ is the wavelength.  If
the sky is assumed to be composed of point sources, then
\begin{equation}\label{eq:antenna-response-point-sources}
    V_{ij}^\text{true}(\nu) = \sum_k a_i(\nu, \hat{r}_k)\,a_j^*(\nu, \hat{r}_k)\,F_k(\nu)\,
        \exp\Big(2\pi i\hat{r}_k\cdot\vec{b_{ij}}/\lambda\Big)\,,
\end{equation}
where $F_k$ is the flux of the $k$th point source in the direction $\hat{r}_k$.

A typical calibration strategy using, for example, the Very Large Array (VLA) involves periodically
pointing at a known compact point source. For a compact point source at the phase center, the phase
of each visibility is zero, and the amplitude is given by the known flux of the source.
Periodically revisiting this source allows for the observer to establish the time variation of the
calibration parameters. Each time the observer solves for the gains that minimize
\begin{equation}
    \chi^2 \propto \sum_{i, j}
        \Big\|V_{ij}^\text{measured}(\nu) - g_i(\nu)\,g_j^*(\nu)\,F\Big\|^2\,,
\end{equation}
where $F$ is the flux of the isolated point source.

This optimization can be performed with rapid convergence using a variant of alternating
least-squares developed by \citet{2008ISTSP...2..707M} and \citet{2014A&A...571A..97S}. At each
iteration this algorithm applies linear least-squares to minimize $\chi^2$ while holding one set of
gains constant:
\begin{equation}\label{eq:stefcal-iterations}
    g_i \leftarrow \frac
        {\sum_{j\neq i} g_j^* V_{ij}^{\text{model},*} V_{ij}^\text{measured}}
        {\sum_{j\neq i} \| g_j V_{ij}^\text{model} \|^2}\,,
\end{equation}
where we have now allowed for a more general sky model to be used during calibration by introducing
the model visibilities $V_{ij}^\text{model}$, which are the true visibilities for an assumed model
of the sky.  Naively applying Equation~\ref{eq:stefcal-iterations} will result in poor convergence
due to oscillations about a minimum of $\chi^2$.  These oscillations can be damped by averaging
subsequent iterations, and \citet{2014A&A...571A..97S} demonstrated that this simple gradient-free
optimization strategy converges remarkably quickly.

The OVRO-LWA is capable of imaging the entire hemisphere in a snapshot image. This brings its own
unique calibration challenges because it is currently impossible to isolate a single compact point source
within the field of view of the interferometer.\footnote{
    Gated pulsar observations could, in principle, achieve this isolation. This capability is a key
    development area for the OVRO-LWA.
}
Due to the wide field of view, determining an accurate gain calibration relies on a detailed sky and
antenna beam model. Mistakes or omissions in the sky model can, for example, generate artificial
ripples in the bandpass that will impact the interferometer's ability to cleanly separate foreground
emission from cosmological 21~cm emission \citep{2016MNRAS.461.3135B, 2017MNRAS.470.1849E}.

Furthermore, at frequencies $\nu < 100\,\text{MHz}$ there are few flux calibrators.
\citet{1977A&A....61...99B} determined the absolute spectrum of Cyg~A between 20~MHz and 2~GHz.
\citet{2012MNRAS.423L..30S} added six additional calibrators, and \citet{2017ApJS..230....7P} used
the VLA 4-band system to bring the total number of available calibrators to 11. However, in
Chapter~\ref{chapter3} I will show that the latter spectra can diverge substantially from truth
when extrapolated below 50~MHz.

Detailed sky and beam models are therefore generally an important calibration requirement for
low-frequency interferometers.  In Chapter~\ref{chapter3}, I will derive an empirical beam model for
the OVRO-LWA and developed a new imaging formalism that captures the entire visible sky in a single
synthesis imaging step that can be used as part of a self-calibration loop
\citep{1978ApJ...223...25R}.

As part of this thesis I developed the \texttt{TTCal} calibration routine for the purpose of
calibrating the OVRO-LWA. It implements the alternating least-squares algorithm described above to
solve for the complex-valued gain of each signal path from a model sky composed of any number of
point sources, Gaussians, and shapelet components \citep{2003MNRAS.338...35R}. If desired,
\texttt{TTCal} may instead solve for the Jones matrix associated with each antenna for a fully
polarized calibration solution. \texttt{TTCal} is freely available under the GPLv3 license or under
any later version.\footnote{
    \url{https://github.com/mweastwood/TTCal.jl}
}

%Some interferometers (e.g., HERA and the MWA), recognizing the difficulty of gain calibration at low
%frequencies, have opted for partly redundant antenna configurations. These configurations can solve
%for many of their calibration parameters internally without relying on an incomplete sky model and
%potentially inaccurate antenna beam model \citep{2010MNRAS.408.1029L}. However, these
%interferometers sacrifice imaging fidelity, which is useful for establishing the remaining
%calibration parameters (e.g., the overall bandpass cannot be solved for in an internal
%redundant-calibration routine).

\section{Source Removal and Direction-Dependent Calibration}

\begin{figure}
    \centering
    \begin{tabular}{ccc}
        \includegraphics[width=0.3\textwidth]{figures/chapter2/cas-a-no-removal} &
        \includegraphics[width=0.3\textwidth]{figures/chapter2/cas-a-subtraction} &
        \includegraphics[width=0.3\textwidth]{figures/chapter2/cas-a-peeling} \\
        (a) & (b) & (c) \\
    \end{tabular}
    \caption{
        Illustration of the improvement in source removal associated with peeling using
        \texttt{TTCal}. (a) Image of Cas~A prior to source removal. (b) Image of Cas~A after
        subtracting a point source without the application of direction-dependent gains. (c) Image
        of Cas~A after peeling (including the application of direction-dependent gains).
    }
    \label{fig:peeling-illustration}
\end{figure}

The large field of view of the OVRO-LWA comes with additional challenges associated with the
ionosphere, inhomogeneous primary beams, and mutual coupling between antennas.

The dispersion relation for electromagnetic waves propagating in a plasma is
\begin{equation}
    \omega^2 = \omega_\text{plasma}^2 + c^2 k^2\,,
\end{equation}
where $\omega$ is the angular frequency of the oscillations, $k$ is the wavenumber, $c$ is the speed
of light, and $\omega_\text{plasma}$ is the plasma frequency---the frequency of electrostatic
oscillations within the plasma. The plasma frequency (in SI units) is given by
\begin{equation}
    \omega_\text{plasma} = \sqrt{\frac{n_e e^2}{m_e \varepsilon_0}}\,,
\end{equation}
where $n_e$ is the number density of electrons, $e$ is the charge of the electron, $m_e$ is the mass
of the electron, and $\varepsilon_0$ is the permittivity of free space.  For Earth's ionosphere, the
plasma frequency is typically $\sim 2\pi \times 10\,\text{MHz}$.  The index of refraction is
\begin{equation}
    n = \sqrt{1 - \frac{\omega_\text{plasma}^2}{\omega^2}}
      \approx 1 - \frac{1}{2}\frac{\omega_\text{plasma}^2}{\omega^2}\,,
\end{equation}
where the approximation holds if $\omega^2 \gg \omega_\text{plasma}^2$. In this regime, the arrival
time of a burst of radio emission is $\propto \nu^{-2}$ as is commonly seen in pulsar astronomy.
The additional phase imparted is, however, $\propto \nu^{-1}$ such that along a given line of sight
the phase can be parameterized as
\begin{equation}
    \phi \approx \phi_0 + \overbrace{2\pi\tau\nu}^\text{delay}
    + \overbrace{\frac{e^2}{4\pi m_e c \varepsilon_0}
        \frac{1}{\nu}\int n_e\,\d l}^\text{ionospheric dispersion}\,,
\end{equation}
where $\phi_0$ sets the overall phase, $\tau$ is the delay, and the integral of the electron number
density along the line of sight is called the Total Electron Content (TEC).

The diffractive scale $r_\text{diff}$ of the ionosphere is the length scale over which the phase
variance is $1\,\text{rad}^2$. Approximately 90\% of the time the diffractive scale is
$>2\,\text{km}$ at 70\,MHz and $>1\,\text{km}$ at 35\,MHz \citep{2016RaSc...51..927M}.  The Fresnel
scale is $r_\text{f} = \sqrt{\lambda D / 2\pi}$ where $D$ is the height of the ionosphere (typically
$\sim 300\,\text{km}$). In the weak scattering regime ($r_\text{diff} \gg r_\text{f}$), the
ionosphere can contribute amplitude and phase scintillation, which may be folded into the antenna
gain calibration.  However, in the strong scattering regime ($r_\text{diff} \lesssim r_\text{f}$),
point sources may become multiply imaged, which cannot be described as a perturbation to the antenna
response \citep{2015MNRAS.453..925V}. Typically the OVRO-LWA operates in the weak scattering regime
with baselines shorter than the diffractive scale of the ionosphere.

A compact interferometer is composed of antennas that are staring through the same patch of the
ionosphere. The ionosphere therefore imparts a phase gradient across the array that refracts sources
from their true position.  In contrast, on longer baselines, the additional phase between the two
antennas may not be correlated \citep{2005ASPC..345..399L}.

However, The antenna response is also generally expected to be inhomogeneous.  Within the core of
the OVRO-LWA, antennas are separated by as little as 5\,m.  \citet{2011ITAP...59.1855E} studied the
impact of mutual coupling on the antenna primary beam within the first Long Wavelength Array station
in New Mexico (LWA1), which uses the same antennas and same 5\,m minimum spacing as the OVRO-LWA,
but the antennas are packed within a 100\,m diameter (as opposed to a 200\,m diameter for the
OVRO-LWA). The authors found that, between 20--74\,MHz, when pointing more than
10$\arcdeg$--20$\arcdeg$ from zenith, mutual coupling and correlated galactic noise led to a
2--6\,dB increase in the system equivalent flux density (SEFD), and 1--2\,dB deviations in the
primary beam pattern between antennas. We expect comparable effects for the OVRO-LWA.

Direction-dependent calibration therefore attempts to account for ionospheric scintillation and
refraction, and inhomogeneous antenna beams by allowing the antenna response in
Equation~\ref{eq:antenna-response-point-sources} to be a free parameter.  \texttt{TTCal} implements
direction-dependent calibration during source subtraction in an algorithm known as peeling
\citep{2008ISTSP...2..707M}. Figure~\ref{fig:peeling-illustration} illustrates the improvement
associated with applying direction-dependent gains during the subtraction of Cas~A---one of the two
brightest point sources in the sky.

\section{Commissioning Challenges}
\label{sec:commissioning-challenges}

\subsection{Computing Antenna Positions}

\begin{figure}
    \centering
    \includegraphics[width=0.7\textwidth]{figures/chapter2/northing-easting-mistake/northing-easting-mistake}
    \caption{
        Illustration of the error in the WCS prior to a correction to the antenna positions. The
        image is a difference between an image constructed with the incorrect antenna positions and
        the corrected antenna positions. The arrow denotes the direction and approximate center of
        the rotation.
    }
    \label{fig:northing-easting-mistake}
\end{figure}

Early images produced by the OVRO-LWA (prior to 2015 October 16) were afflicted by an apparent
rotation in the World Coordinate System (WCS). This rotation is illustrated in
Figure~\ref{fig:northing-easting-mistake}.

When data is streamed to the ASTM, it arrives in a raw, unordered format specific to the operation
of the correlator. The very first step in any analysis is to convert from this format into the
standard \texttt{MeasurementSet} format, which is used by, for example, the National Radio Astronomy
Observatory's (NRAO) Common Astronomy Software Applications (CASA) package. This data conversion
step is performed by the \texttt{dada2ms} program written by Stephen Bourke. As part of this
conversion process, \texttt{dada2ms} computes and attaches additional metadata such as the antenna
positions, frequency of the observations, and the direction of the phase center. Unfortunately an
error had been made in the calculation of the antenna positions.

While many astronomers are familiar with a wide range of celestial coordinate systems, Earth
coordinate systems are somewhat more esoteric. In particular, there are differences between geodetic
systems that seek to describe positions across the entire Earth and those that only seek to describe
positions on a single continental plate. The former geodetic systems are useful as they specify an
absolute position on the surface of the Earth, while the latter geodetic systems are insensitive to
continental drift and therefore do not naturally change with time. The OVRO-LWA antenna positions
were surveyed in the geodetic system specified by the North American Datum of 1983 (NAD\,83), and
reported in the Universal Transverse Mercator (UTM) coordinate system. NAD\,83 was designed to
closely match the World Geodetic System of 1984 (WGS\,84), and the difference between the two is
generally too small to be of concern to long wavelength radio astronomy (the error in using them
interchangeably is of order one meter in the absolute position of the interferometer).

The UTM coordinate system is described by the coordinate values northing and easting, each measured
in units of meters.  The UTM coordinate system is designed to be a square grid on the surface of a
sphere. This is in contrast to the more familiar latitude and longitude, where a change in longitude
corresponds to a smaller physical distance at high latitudes. Instead, a 1\,km change in easting
corresponds to a physical distance of approximately 1\,km regardless of the position of the
measurement. Consequently, a line of constant easting cannot run true north. Initial calculations of
the OVRO-LWA antenna positions had erroneously assumed that northing runs true north, and easting
runs true east.\footnote{
    Incidentally, the OVRO-LWA is not the first (or the last) interferometer to fall victim to this
    pitfall. The MWA suffered from this mistake during commissioning, and the CHIME pathfinder was
    mistakenly built with its cylindrical focusing surfaces slightly rotated from true north (as was
    the intention).
}

All told, the impact of this mistake was an erroneous $\sim 1\arcdeg$ rotation about zenith in the
antenna positions. Cyg~A and Cas~A, as the brightest point sources in the northern hemisphere at low
radio frequencies, are used to derive the phase calibration of the interferometer. Because the
antenna positions were erroneous, the phase calibration attempts to correct the position of Cyg~A
and Cas~A by applying a phase gradient across the array such that the two sources match their
catalog positions as closely as possible. Images produced by the interferometer therefore appeared
to be rotated by $\sim 1\arcdeg$ about a position roughly between the location of Cyg~A and Cas~A
during calibration. This offset of the rotation center can be seen in
Figure~\ref{fig:northing-easting-mistake}.

I fixed the erroneous calculation of antenna positions by patching \texttt{dada2ms} to stop relying
on the assumption that northing runs true north and easting runs true east. After applying this
patch, \texttt{dada2ms} now correctly converts from the surveyed NAD\,83 UTM coordinate values to
WGS\,84 longitude--latitude values, and finally to the International Terrestrial Reference Frame
(ITRF) using a conversion routine provided by the \texttt{casacore} software package.  In addition,
I discovered a similar coordinate rotation in sky maps generated by the first Long Wavelength Array
(LWA1) station located in New Mexico \citep{2017MNRAS.469.4537D}. Working in collaboration with the
authors of those sky maps, the corrected LWA1 sky maps are now publicly available online.

\subsection{Frequency Channel Labeling}

After correcting the calculation of antenna positions, we achieved improved agreement between
apparent source positions and their catalog positions. However, there was still a residual
systematic error apparent in images constructed during 2015 December. In these images, sources
appeared to be radially offset from the position of Cyg~A and Cas~A.

The ionosphere is a natural culprit because refraction due to propagation through the ionosphere
will tend to move sources to higher elevation \citep[e.g.,][]{2014MNRAS.437.1056V}. Confusingly, the
sense of the observed source offsets was opposite to the expectation of the ionosphere. Sources
appeared at lower elevations than expected. It appeared as if the length of each baseline was
somehow 0.3\% longer than expected.

Such an error could arise due to a mistake in the survey of the antenna positions or another error
in the calculation of their ITRF coordinates. Alternatively, because the action of an interferometer
is dependent on the ratio of the baseline length to the wavelength, the error could be generated by
a mislabeling of the frequency channels output by the correlator. I found that a frequency offset of
$\sim 100\,\text{kHz}$ could be enough to explain the residual source offsets seen in Cyg~A and
Cas~A. Ryan Monroe later confirmed that the error was in the correlator by examining the frequency
of an FFT artifact that arises from the sampling frequency and therefore is a known spectral
feature. This feature was offset from its true location by $\approx150\,\text{kHz}$, and therefore
accounted for the radial offset seen in the source positions.

\subsection{RFI Localization}

\begin{figure}
    \centering
    \begin{tabular}{cc}
        \includegraphics[width=0.45\textwidth]{figures/chapter2/google-maps-rfi-localization} &
        \includegraphics[width=0.45\textwidth]{figures/chapter2/power-line-picture} \\
        (a) & (b) \\
    \end{tabular}
    \caption{
        (a) The localization region (roughly 100\,m by 1.5\,km) for a source of RFI south of the
        OVRO-LWA and near the town of Big Pine. Satellite imagery \textcopyright2018 Google. Map
        data \textcopyright2018 Google.
        (b) Image of a high-voltage power line overlooking OVRO near the localization region.
    }
    \label{fig:rfi-localization}
\end{figure}

An ongoing challenge faced by the OVRO-LWA is the presence of broadband sources of radio frequency
interference (RFI) in the vicinity of the observatory. Due to the entire-hemisphere field of view of
the OVRO-LWA, these sources appear as points on the horizon that limit the sensitivity of snapshot
images through additional sidelobe noise. Further complicating matters, because this RFI originates
from the horizon, the antennas shadow each other leading to an unpredictable antenna responses in
the direction of each source. This impedes traditional deconvolution techniques.  Fortunately,
because these sources are typically in the near-field of the interferometer, the curvature of the
incoming wavefront can be used to infer the distance to each source of RFI.

The path difference from a source in the near-field of an interferometer located at the position
$(\xi, \eta, \zeta)$ to two antennas located respectively at $(x_i, y_i, z_i)$ and $(x_j, y_j, z_j)$
is
\begin{equation}\label{eq:nearfield-path-difference}
    \Delta l^\text{near-field}_{ij} =
        \sqrt{(x_j - \xi)^2 + (y_j - \eta)^2 + (z_j - \zeta)^2}
        - \sqrt{(x_i - \xi)^2 + (y_i - \eta)^2 + (z_i - \zeta)^2}\,.
\end{equation}
In the limit that the distance of the source goes to infinity, we recover the familiar expression
\begin{equation}\label{eq:farfield-path-difference}
    \Delta l^\text{far-field}_{ij} = \frac{1}{D}\Big(
        (x_i - x_j)\,\xi + (y_i - y_j)\,\eta + (z_i - z_j)\,\zeta
    \Big)\,,
\end{equation}
where $D$ is the distance to the source. The correlation measured between two antennas for a source
in the near-field of the interferometer is therefore
\begin{equation}\label{eq:nearfield-visibilities}
    V_{ij} = F \exp\Big(2\pi i \Delta l^\text{near-field}_{ij}/\lambda\Big)\,,
\end{equation}
where $V_{ij}$ is the visibility measured between antennas $i$ and $j$, $F$ is the apparent
brightness of the source, and $\lambda$ is the wavelength.

In 2016 May, I used Equation~\ref{eq:nearfield-visibilities} to estimate the position of the four
sources of RFI at 67\,MHz. The brightest of these sources can be seen in the lower-right corner of
Figure~\ref{fig:northing-easting-mistake}, and its localization can be seen in
Figure~\ref{fig:rfi-localization}. This work was instrumental in identifying faulty insulators on
high-voltage power lines as the source of pulsed broadband RFI. While this particular source of RFI
has now been repaired, we are working with the Los Angeles Department of Water and Power (LADWP) to
identify and repair the remaining RFI sources.

\subsection{Polarization Swaps}

While performing maintenance on antennas, occasionally the signal paths corresponding to the $x$ and
$y$ dipoles would be carelessly swapped.\footnote{
    The author of this thesis accepts responsibility for some---but not all---of these events!
}
After correlation, this would lead to some $xx$ and $yy$ correlations being mislabeled as $xy$ and
$yx$ correlations, and vice versa. This is clearly a problem for polarized imaging, because it
allows unpolarized emission to spill into the polarized images. Similar errors are produced in
unpolarized images, but the fractional error is less due to the fact that most of the sky emission
is unpolarized at low frequencies.

Marin Anderson identified a simple metric that allows for the rapid identification of antennas with
polarizations swapped. That is, if the amplitude for most baselines involving a given antenna have
the property that the cross-polarization visibilities are higher amplitude than the co-polarization
visibilities then it can be said with high confidence that this antenna has a polarization swap. I
built a tool that relabels the polarizations in datasets with a known set of ``swapped antennas.''

\subsection{Gain Fluctuations}

\begin{figure}
    \centering
    \includegraphics[width=\textwidth]{figures/chapter2/sawtooth/sawtooth}
    \caption{
        Measurement of the ``sawtooth'' fluctuations in the receiver gains associated with
        temperature variations within the electronics shelter. Four antenna traces are shown here to
        demonstrate that the gain variations are coherent between signal paths.
    }
    \label{fig:sawtooth-gains}
\end{figure}

The OVRO-LWA's receivers are located within a temperature controlled shelter. During typical
operation, the air conditioning system cycles on a 15--17\,minute timescale. The action of this is
that the temperature within shelter varies with a 15--17\,minute period. The total amplification
within the analog signal path is temperature sensitive and varies by $\sim 0.1\,\text{dB}$ within a
cycle. These gain fluctuations are illustrated in Figure~\ref{fig:sawtooth-gains}.

Typically, the complex gain calibration described in \S\ref{sec:gain-calibration} is performed once
per day, and therefore does not account for the gain fluctuations associated with these temperature
fluctuations. We add an additional stage of gain calibration to account for these time fluctuations.
The amplitude of each antenna's auto-correlation is smoothed on a 45\,minute timescale to remove the
contribution of the sawtooth pattern to each auto-correlation. The ratio of the smoothed
auto-correlations to the original auto-correlation defines a per-antenna correction that is then
applied to the cross-correlations, removing the amplitude fluctuations with respect to time. This
procedure, does not account for any fluctuations in the phase with respect to time.

In principle, one could account for gain fluctuations (amplitude and phase) by recalibrating more
frequently than once per day. Ideally, one might even like to critically sample the sawtooth
fluctuations seen in Figure~\ref{fig:sawtooth-gains}. In practice, however, this is difficult due to
the availability of strong calibrator sources (if Cyg~A and Cas~A are at low elevations or below the
horizon, calibration is difficult), and the need to use $\gtrsim 10$\,minutes of data during
calibration to avoid ionospheric fluctuations impacting the gain solution. Future development of the
OVRO-LWA should record the temperature outside near the antennas, and near the analog receivers to
aid in calibrating these gain fluctuations.

\subsection{Common-Mode RFI}

With the current analog signal path of the OVRO-LWA there appears to be an additive component to the
measured visibilities (see Figure~\ref{fig:fitrfi} for images of the contribution to snapshot
images). While the impact of this apparent common-mode RFI will be discussed in more detail in
\S\ref{sec:rfi}, we will briefly summarize how this is mitigated here.

The operating principle is that terrestrial sources of RFI are not attached to the sky and therefore
do not sweep through the fringe pattern of the interferometer at the sidereal rate. Instead these
sources can be at a fixed position whether the interference enters through the antennas or somehow
couples into the analog signal path. Therefore, by simply averaging the measured correlations over a
period of 24\,hr, the contribution of true sky emission is smeared over tracks of constant
declination. Persistent sources of RFI, however, will generally add coherently.

We identify pairs of antennas that are especially susceptible to common-mode RFI by comparing the
amplitude of each correlation after averaging to other baselines of similar length. This measurement
essentially constrains the degree to which a correlation is washed out by time averaging. A
correlation that does not appear to drop in amplitude after averaging is assumed to carry a large
component due to common-mode RFI.  Antenna pairs with adjacent signal paths within the receiver tend
to register as outliers, which is suggestive of some amount of cross-coupling between the signal
paths. However, this is not a complete explanation and physical proximity of the signal paths is not
a requirement for a correlation to be dominated by common-mode RFI.

In addition to flagging these baselines, the dominant components of the time-averaged visibilities
are taken as models for the RFI contribution to the visibilities. The model of each RFI component is
then scaled and removed from each integration to help suppress the degree of contamination. This
process is somewhat successful in removing ring-like artifacts in long synthesis images (see
Figure~\ref{fig:rings}), but ongoing development at the OVRO-LWA will see the replacement of the
analog receivers, which will obviate the need for the modeling and removal of this pickup.

\myputbib{thesis}
\end{bibunit}

