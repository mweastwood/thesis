\cleartoevenpage

\myepigraph
{There's never enough time to do all the nothing you want.}
{Bill Watterson}

\chapter{Conclusion and Future Outlook}
\label{chapter5}

\begin{bibunit}

The 21\,cm transition of neutral hydrogen promises to open a new window into the high redshift
universe---the EoR and the Cosmic Dawn. The 21\,cm brightness temperature is a probe of the spin
temperature, density, and ionization state of the IGM.  A detection of this transition will allow
astronomers to probe the first generation of star and galaxy formation through their influence on
the IGM. Because of the wide field of view of low-frequency radio telescopes, and the fact that the
observed redshift of the transition can be used to measure the radial distance, future low frequency
radio telescopes may be able to produce tomographic maps of the high redshift universe.

The current generation of radio telescopes, however, must restrict themselves to statistical
measurements due to sensitivity considerations: usually the global average, and the spatial power
spectrum.

During the EoR, fluctuations in the 21\,cm brightness
temperature are primarily sourced by inhomogeneous ionization and the patchy reionization process.
During the Cosmic Dawn, fluctuations are instead sourced by inhomogeneous heating and Ly$\alpha$
emission.

In this thesis I have described the construction and commissioning of the OVRO-LWA.  At times this
involved untangling hundreds of 10\,m cables. At other times this involved untangling dozens of
1\,km fiber optic cables before literally hauling them through the desert sand and back into the
conduit.  Sometimes the desert wind made it impossible to make clean fiber optic connections without
eating a face full of dirt. On occasion it involved driving around the neighborhood with an
antenna mounted on top of the car only to be threatened by some locals.\footnote{
    They turned out to be friendly, and we got to share with them our RFI hunting apparatus.
}

As part of the commissioning of the OVRO-LWA, I developed several tools that have allowed the
instrument to capture high-dynamic range, high-fidelity images.

Notably, I wrote the \texttt{TTCal} calibration routine that is used to calibrate the interferometer
and remove point sources from the visibilities.

In Chapter~\ref{chapter3}, I described the development and demonstration of a new widefield imaging
technique specialized for drift-scanning interferometers like the OVRO-LWA. Tikhonov-regularized
$m$-mode analysis imaging constructs an image of the entire sky in a single synthesis imaging step
without the need for gridding. I produced eight 

Foreground mitigation and instrumental calibration are the limiting factors in essentially all
attempts to measure the spatial power spectrum of 21\,cm fluctuations from the EoR and Cosmic Dawn.

In Chapter~\ref{chapter4}, I placed the deepest limits to date on the amplitude of 21\,cm
fluctuations from the Cosmic Dawn, and the first limits at $z > 18$.

This measurement was the first demonstration of the double KL transform foreground filter on a real
measured dataset. I showed that while this filter can help to mitigate foreground contamination, it
is not a replacement for careful calibration. Calibration and modeling errors can allow foreground
emission to leak through the foreground filter.

This measurement, however, was limited by systematic errors. These systematic errors appear to arise

In order to make a successful and convincing detection of the 21\,cm power spectrum, the calibration
requirements are strict. The overall bandpass of the interferometer must be known to better than
0.01\%, and the gain errors on each antenna must be less than 0.1\%. The OVRO-LWA was designed with
the goal of capturing high fidelity snapshot images, and therefore the baseline configuration is
maximally non-redundant.

Other experiments have opted for the opposite approach of a maximally redundant configuration (e.g.,
PAPER, HERA, and the upgraded MWA). One advantage of this alternate approach is the application of
redundant calibration \citep{2010MNRAS.408.1029L}. Redundant calibration solves for the $N$
calibration parameters by using redundant information encoded within the $N^2$ visibilities, and
demanding that baselines with the same length and orientation must produce the same output (to
within thermal noise). Redundant calibration largely sidesteps problems associated with incomplete
sky models during calibration because no sky model is required. However, ultimately there are some
calibration parameters that cannot be obtained through redundant calibration---notably the overall
bandpass. The overall bandpass is therefore still generally obtained from observations of the sky,
and a maximally redundant interferometer is not optimally configured for imaging.

The question of whether or not to pursue a redundant array configuration for calibration purposes is
therefore not as clear cut as it may initially seem.  Indeed a hybrid approach may be necessary.
\citet{2010MNRAS.408.1029L} and \citet{2017arXiv170101860S} describe how redundant calibration may
be generalized to include some sky information.  It may also be advantageous to have, an imaging
interferometer could construct sky models that a co-located redundant interferometer can use to
establish its bandpass. However, ultimately the imaging interferometer may have an easier time
proving that its bandpass is actually calibrated at the level of 0.01\%.

Better peeling

There is a growing need to map the response of each individual dipole.

In this thesis I mapped the array-averaged beam by tracing the apparent flux of several bright point
sources as they passed through the beam. However, without a known flux scale, I exploited symmetry
considerations that broke the degeneracy between beam amplitude and source flux \citep[inspired
by][]{2012AJ....143...53P}. This process can be repeated across the bandwidth of the instrument, but
is limited by the density of suitably bright point sources, and ionospheric scintillation, which
dominates the noise in the measurement. In principle, this work could be extended to map the
individual dipole beams through the use of direction-dependent calibration
\citep{2008ISTSP...2..707M}, but this will limit even further the number of suitably bright sources.
Ultimately, any assumptions about the symmetry of the beam are unsatisfactory, because once the
antenna is embedded in the interferometer with other antennas located in close proximity, none of
these assumed symmetries will continue to hold.

A better approach is to correlate each element of the interferometer with a large steerable antenna,
such as the OVRO 40\,m antenna \citep[e.g.,][]{2016SPIE.9906E..0DB}. In this measurement the
steerable tracks a set of bright point sources through the sky. The amplitude of the correlation is
therefore a measure of the response of each antenna in the direction of the point source.
Furthermore, the steerable antenna can be used to set the flux scale without the need to assume a
functional form for the beam model. One drawback to this approach, however, is the presence of other
emission in the sky that will also contribute to the measured correlation.

A final possibility is to mount transmitter to the underside of a drone and make several passes over
the array. This approach has been attempted by \citet{2017PASP..129c5002J} to map the response of a
single dipole, but was limited by the stability of the drone. Future efforts to use drones to map
the antenna response will likely need to focus on careful measurement of the transmitting antenna
beam and the real time position and orientation of the drone. However, this approach is advantageous
in that it can map the array elements in-situ including mutual coupling effects that perturb the
antenna response.

The MWA has also been able to map their beam models using the transmission from a satellite
\citep{2018arXiv180804516L}, finding reasonable ($\sim 1\,\text{dB}$) agreement with fully embedded
element models \citep{2017PASA...34...62S}. While a satellite can only be used to map the antenna
response within a narrow bandwidth, this technique has value in validating increasingly
sophisticated models of the primary beam.

\myputbib{thesis}
\end{bibunit}

