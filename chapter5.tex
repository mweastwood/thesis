\cleartoevenpage

\myepigraph
{There's never enough time to do all the nothing you want.}
{Bill Watterson}

\chapter{Conclusion and Future Outlook}
\label{chapter5}

\begin{bibunit}

The 21\,cm transition of neutral hydrogen promises to open a new window into the high redshift
universe---the EoR and the Cosmic Dawn. The 21\,cm brightness temperature is a probe of the spin
temperature, density, and ionization state of the IGM.  A detection of this transition will allow
astronomers to probe the first generation of star and galaxy formation through their influence on
the IGM. Because of the wide field of view of low-frequency radio telescopes, and the fact that the
observed redshift of the transition can be used to measure the radial distance, future low frequency
radio telescopes may be able to produce tomographic maps of the high redshift universe.

The current generation of radio telescopes, however, must restrict themselves to statistical
measurements due to sensitivity considerations: usually the global average, and the spatial power
spectrum. The first detection of the globally averaged 21\,cm transition was reported by
\citet{2018Natur.555...67B} at 78\,MHz and has generally lead to more questions than answers. The
power spectrum, which measures the amplitude of fluctuations in the 21\,cm brightness temperature
could be a corroborating and complementary measurement.  During the EoR, fluctuations in the 21\,cm
brightness temperature are primarily sourced by inhomogeneous ionization and the patchy reionization
process.  During the Cosmic Dawn, fluctuations are instead sourced by inhomogeneous heating and
Ly$\alpha$ emission.

In this thesis I have described the construction and commissioning of the OVRO-LWA, a new low
frequency radio telescope (28--85\,MHz).  At times this involved untangling hundreds of 10\,m
cables. At other times this involved untangling dozens of 1\,km fiber optic cables before literally
hauling them through the desert sand and back into the conduit.  Sometimes the desert wind made it
impossible to make clean fiber optic connections without eating a face full of dirt. On occasion it
involved driving around the neighborhood with an antenna mounted on top of the car only to be
threatened by some locals.\footnote{
    They turned out to be friendly, and we got to share our RFI hunting apparatus with them. In this
    particular encounter we discovered that fluorescent lights can be a source of RFI, because these
    residents happened to turn off their garage lights as we drove by.
}
However, the OVRO-LWA is now generating its first scientific results.

As part of the commissioning of the OVRO-LWA, I developed several tools that have allowed the
instrument to capture high-dynamic range, high-fidelity images.  In particular, I wrote the open
source \texttt{TTCal} calibration routine that is used to both calibrate the interferometer and
remove point sources from the visibilities. The use of \texttt{TTCal} for peeling and
direction-dependent calibration was instrumental for the work presented in this thesis as well as a
search for low-frequency radio emission associated with a gamma ray burst
\citep{2017arXiv171106665A}.

In Chapter~\ref{chapter3}, I described the development and demonstration of a new widefield imaging
technique specialized for drift-scanning interferometers like the OVRO-LWA. Tikhonov-regularized
$m$-mode analysis imaging constructs an image of the entire sky in a single synthesis imaging step
without the need for gridding. I adapted the CLEAN algorithm to deconvolve the point spread function
of the generated sky maps, produced the eight highest angular resolution maps of the entire sky
above $\delta=-30\arcdeg$ below 100\,MHz. These sky maps are intended to be used to model and
mitigate foreground contamination in the detection of the high redshift 21\,cm transition, and are
therefore now freely available on LAMBDA.

Foreground mitigation and instrumental calibration are the limiting factors in essentially all
attempts to measure the spatial power spectrum of 21\,cm fluctuations from the EoR and Cosmic Dawn.
The sensitivity of a given instrument to the 21\,cm power spectrum is therefore primarily a function
of how well the experiment can control these systematic errors. This has led to the development of a
plethora of strategies for dealing with foreground contamination: delay filters, modeling and
subtraction, the SVD, and GMCA. Delay filters assume that the foreground emission (after application
of the instrumental response) is spectrally smooth. Modeling and subtraction is exceptionally
laborious and can introduce a dangerously large number of non-linear free parameters. The SVD and
GMCA are blind foreground removal techniques that do not build in any knowledge of the signal or
contaminating foreground emission.

In Chapter~\ref{chapter4} I demonstrated for the first time on a measured dataset the application of
the double KL transform foreground filter described by \citep{2014ApJ...781...57S,
2015PhRvD..91h3514S}, which exploits information contained in the full covariance matrix of the
dataset to optimally separate foreground emission from the 21\,cm signal. I measured the system
temperature of the OVRO-LWA, and the angular covariance of the foreground emission. Then with a
fiducial model of the 21\,cm power spectrum I computed the corresponding foreground filter and
applied it to a dataset from the OVRO-LWA.  I showed that while this filter can help to mitigate
foreground contamination, it is not a replacement for careful calibration. Errors in calibration and
modeling can allow foreground emission to leak through the foreground filter.

In Chapter~\ref{chapter4}, I applied the double KL transform foreground filter to place the deepest
limits to date on the amplitude of 21\,cm fluctuations from the Cosmic Dawn, and the first limits at
$z > 18$. While these upper limits do not currently constrain models of early star formation, they
represent an important step towards that goal. I found that this measurement was likely limited by
the quality of the calibration of the interferometer.  In order to make a successful and convincing
detection of the 21\,cm power spectrum, the calibration requirements are strict. The overall
bandpass of the interferometer must be known to better than 0.01\%, and the gain errors on each
antenna must be less than 0.1\%. The OVRO-LWA was designed with the goal of capturing high fidelity
snapshot images, and therefore the baseline configuration is maximally non-redundant. This design
choice shapes the calibration strategies that are possible with the interferometer.

Other experiments have opted for the opposite approach of a maximally redundant configuration (e.g.,
PAPER, HERA, and the recently expanded MWA). One advantage of this alternate approach is the
application of redundant calibration \citep{2010MNRAS.408.1029L}. Redundant calibration solves for
the $N$ calibration parameters by using redundant information encoded within the $N^2$ visibilities,
and demanding that baselines with the same length and orientation must have the same correlation (to
within thermal noise). Redundant calibration largely sidesteps problems associated with incomplete
sky models during calibration because no sky model is required. However, ultimately there are some
calibration parameters that cannot be obtained through redundant calibration---notably the overall
bandpass. The overall bandpass is therefore still generally obtained from observations of the sky,
and a maximally redundant interferometer is not optimally configured for imaging.

The question of whether or not to pursue a redundant array configuration for calibration purposes is
therefore not as clear cut as it may initially seem.  Indeed a hybrid approach may be necessary.
\citet{2010MNRAS.408.1029L} and \citet{2017arXiv170101860S} describe how redundant calibration may
be generalized to include some sky information.  It may also be advantageous to have, an imaging
interferometer construct sky models that a co-located redundant interferometer can use to establish
its bandpass. However, ultimately the imaging interferometer may have an easier time proving that
its bandpass is actually calibrated at the level of 0.01\% through observations of flux
calibrators.\footnote{
    Current low frequency flux calibrators are typically only known at the level of 5--10\%, so
    additional work is needed there too.
}

A surprising result seen in Chapter~\ref{chapter4} was that peeling and point source removal didn't
necessarily improve the sensitivity of the measurement. This is likely a reflection of frequency
dependent errors introduced by the stationary component removal or source removal routines.
Development work is ongoing to replace the OVRO-LWA's analog receivers, which should obviate the
need for stationary component removal.  A likely perpetrator is \texttt{TTCal}'s peeling routine,
which currently operates independently on each channel and introduces a large number of free
parameters. Although peeling leads to a substantial improvement in sidelobe levels for snapshot and
$m$-mode analysis images, its use for 21\,cm cosmology should include additional regularization that
forces the solution to be spectrally smooth.

Peeling is necessary to account for ionospheric effects that cannot be known a priori. However, at
the moment peeling is used to additionally account for the variance in primary beam shapes within
the interferometer. In principle, these can be known a priori, which would reduce the number of
necessary free parameters in a parameterization of the direction-dependent calibrations.

In this thesis I mapped the array-averaged beam by tracing the apparent flux of several bright point
sources as they passed through the beam. However, without a known flux scale, I exploited symmetry
considerations that broke the degeneracy between beam amplitude and source flux \citep[inspired
by][]{2012AJ....143...53P}. This process can be repeated across the bandwidth of the instrument, but
is limited by the density of suitably bright point sources, and ionospheric scintillation, which
dominates the noise in the measurement. In principle, this work could be extended to map the
individual dipole beams through the use of direction-dependent calibration
\citep{2008ISTSP...2..707M}, but this will limit even further the number of suitably bright sources.
Ultimately, any assumptions about the symmetry of the beam are unsatisfactory, because once the
antenna is embedded in the interferometer with other antennas located in close proximity, none of
these assumed symmetries will continue to hold.

A better approach is to correlate each element of the interferometer with a large steerable antenna,
such as the OVRO 40\,m antenna \citep[e.g.,][]{2016SPIE.9906E..0DB}. In this measurement the
steerable tracks a set of bright point sources through the sky. The amplitude of the correlation is
therefore a measure of the response of each antenna in the direction of the point source.
Furthermore, the steerable antenna can be used to set the flux scale without the need to assume a
functional form for the beam model. One drawback to this approach, however, is the presence of other
emission in the sky that will also contribute to the measured correlation. This can be mitigated
through gated observations of pulsars, but additional integration time will then be needed due to
the reduced flux.

A final possibility is to mount transmitter to the underside of a drone and make several passes over
the array. This approach has been attempted by \citet{2017PASP..129c5002J} to map the response of a
single dipole, but was limited by the stability of the drone. Future efforts to use drones to map
the antenna response will likely need to focus on careful measurement of the transmitting antenna
beam and the real time position and orientation of the drone. However, this approach is advantageous
in that it can map the array elements in-situ including mutual coupling effects that perturb the
antenna response.  The MWA has also been able to map their beam models in-situ using the
transmission from a satellite \citep{2018arXiv180804516L}, finding reasonable ($\sim 1\,\text{dB}$)
agreement with fully embedded element models \citep{2017PASA...34...62S}. While a satellite can only
be used to map the antenna response within a narrow bandwidth, this technique has value in
validating increasingly sophisticated models of the primary beam.

This thesis lays the foundation for efforts to open up the 21\,cm window into the Cosmic Dawn with
the OVRO-LWA.  Continued detailed and careful work to characterize and calibrate the interferometer
will be needed. The gain calibration, removal of point sources, and per-antenna primary beam models
are three areas likely to lead to substantial improvements in sensitivity.

\myputbib{thesis}
\end{bibunit}

