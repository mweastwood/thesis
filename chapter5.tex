\chapter{Conclusion}
\label{chapter5}

In this thesis I have described the construction and commissioning of the OVRO-LWA.

Construction of the OVRO-LWA

As part of the commissioning of the OVRO-LWA, I developed several tools that have allowed the
instrument to capture high-dynamic range, high-fidelity images.

I wrote \texttt{TTCal}

In Chapter~\ref{chapter3}, I described the development and demonstration of a new widefield imaging
technique specialized for drift-scanning interferometers like the OVRO-LWA.

Foreground mitigation and instrumental calibration are the limiting factors in essentially all
attempts to measure the spatial power spectrum of 21\,cm fluctuations from the EoR and Cosmic Dawn.

In Chapter~\ref{chapter4}, I placed the deepest limits to date on the amplitude of 21\,cm
fluctuations from the Cosmic Dawn, and the first limits at $z > 18$.

This measurement was the first demonstration of the double KL transform foreground filter on a real
measured dataset. I showed that while this filter can help to mitigate foreground contamination, it
is not a replacement for careful calibration. Calibration and modeling errors can allow foreground
emission to leak through the foreground filter.

This measurement, however, was limited by systematic errors.

Future requirements and directions

Better calibration

Better peeling

Beam mapping

